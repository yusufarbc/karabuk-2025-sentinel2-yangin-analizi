% !TeX program = pdflatex
\documentclass[11pt,a4paper]{article}
\usepackage[T1]{fontenc}
\usepackage[utf8]{inputenc}
\usepackage{lmodern}
\usepackage{microtype}
\usepackage[main=turkish, english]{babel}
\usepackage{geometry}
\geometry{margin=2.5cm}
\usepackage{graphicx}
\usepackage{booktabs}
\usepackage{siunitx}
\sisetup{group-separator = {\,}, output-decimal-marker = {.}}
\usepackage[unicode]{hyperref}
\PassOptionsToPackage{hyphens}{url}
\hypersetup{
  colorlinks=true,
  linkcolor=black,
  citecolor=black,
  urlcolor=blue,
  pdftitle={Karabük 2025 Orman Yangınları: Sentinel-2 ile Uzaktan Algılama Analizi},
  pdfauthor={Yusuf Talha ARABACI}
}
\usepackage{caption}
\usepackage{subcaption}
\usepackage{amsmath}
\usepackage{float}
\usepackage[section]{placeins}

\title{Karabük 2025 Orman Yangınları: Sentinel-2 ve Spektral İndeksler ile Uzaktan Algılama Analizi}
\author{Yusuf Talha ARABACI\\Karabük Üniversitesi\\Yüksek Lisans, Yazılım Mühendisliği Öğrencisi}
\date{08 Aralık 2025}

\begin{document}
\selectlanguage{turkish}
\sloppy
\maketitle
\thispagestyle{empty}

\begin{abstract}
Bu çalışma, 2025 yaz sezonunda Karabük genelinde meydana gelen orman yangınlarının çevresel etkilerini ve yanan alanların mekânsal dağılımını Sentinel-2 uydu verilerini kullanarak analiz etmektedir. Çalışmada, yangın öncesi (Temmuz başı) ve yangın sonrası (Eylül sonu) dönemlere ait görüntüler Google Earth Engine (GEE) platformunda işlenmiş; Normalize Edilmiş Fark Vejetasyon İndeksi (NDVI) ve Normalize Edilmiş Fark Yanma Oranı (NBR) metrikleri kullanılmıştır. Elde edilen dNDVI ve dNBR fark haritaları üzerinden hasar şiddeti sınıflandırılmıştır. Analiz sonuçları, sezonun en yıkıcı yangını olan Büyük Ovacık yangını başta olmak üzere, Eflani ve Toprakcuma bölgelerindeki yangınların ekolojik etkisini sayısal olarak ortaya koymaktadır.
\end{abstract}

\noindent\textbf{Anahtar Kelimeler:} Sentinel-2, Orman Yangını, Karabük, dNBR, dNDVI, Uzaktan Algılama.

\clearpage
\tableofcontents
\clearpage

\section{Giriş}
Orman yangınları, iklim değişikliği ve insan faktörleri ile birlikte artan sıklıkta ve şiddette meydana gelen küresel bir çevre sorunudur. Karabük ili, zengin orman varlığı ve biyoçeşitliliği ile Türkiye'nin önemli ekolojik bölgelerinden biridir. 2025 yaz sezonunda il genelinde, özellikle Ovacık, Eflani ve Safranbolu ilçelerinde birbiri ardına meydana gelen yangınlar, geniş orman alanlarının kaybına neden olmuştur. Bu çalışma, söz konusu yangınların izlerini uzaktan algılama teknolojileri ile tespit etmeyi ve hasar derecesini haritalamayı amaçlamaktadır.

\section{Materyal ve Yöntem}

\subsection{Çalışma Alanı ve Veri}
Çalışma, Karabük il sınırları içerisindeki ormanlık alanları kapsamaktadır. Analiz için Avrupa Uzay Ajansı'nın (ESA) Sentinel-2 uydusuna ait L2A (Atmosferik düzeltilmiş) görüntüleri kullanılmıştır.
\begin{itemize}
    \item \textbf{Öncesi Dönem (Healthy Baseline):} 01 Temmuz 2025 - 20 Temmuz 2025
    \item \textbf{Sonrası Dönem (Post-Fire):} 05 Eylül 2025 - 30 Eylül 2025
\end{itemize}

\subsection{Yöntem}
Görüntü işleme adımları Google Earth Engine (GEE) üzerinde Python API kullanılarak gerçekleştirilmiştir:
\begin{enumerate}
    \item \textbf{Bulut ve Gölge Maskeleme:} QA60 bandı kullanılarak bulutlu pikseller analiz dışı bırakılmıştır.
    \item \textbf{Arazi Örtüsü Maskeleme:} ESA WorldCover 2021 haritası kullanılarak sadece orman (Sınıf 10) ve çalılık (Sınıf 20) alanlar analize dahil edilmiş; tarım arazileri ve yerleşim yerleri maskelenmiştir.
    \item \textbf{İndeks Hesaplamaları:}
    \begin{itemize}
        \item \textbf{NDVI (Bitki Örtüsü İndeksi):} $ (NIR - Red) / (NIR + Red) $
        \item \textbf{NBR (Yanma Oranı İndeksi):} $ (NIR - SWIR) / (NIR + SWIR) $
    \end{itemize}
    \item \textbf{Fark Analizi (Change Detection):}
        $$ dNDVI = NDVI_{sonra} - NDVI_{once} $$
        $$ dNBR = NBR_{once} - NBR_{sonra} $$
\end{enumerate}

\section{Bulgular ve Tartışma}

\subsection{İl Geneli Değerlendirme}
İl genelinde yapılan taramada (Overview), özellikle güneybatı ve kuzeydoğu hatlarında belirgin yanma izlerine rastlanmıştır. Toplam yanan alanın yaklaşık 17.000 hektar düzeyinde olduğu tahmin edilmekle birlikte, yüksek şiddetli yanmanın (High Severity) lokalize olduğu görülmüştür.

\subsection{Bölgesel Yangın Analizleri}
Analiz sonucunda 7 kritik yangın bölgesi detaylı olarak incelenmiştir. Tablo \ref{tab:yanginlar}, bu bölgelere ait temel bulguları özetlemektedir.

\begin{table}[H]
\centering
\caption{2025 Karabük Yangınları Detaylı Analiz Sonuçları}
\label{tab:yanginlar}
\resizebox{\textwidth}{!}{%
\begin{tabular}{@{}llp{5cm}l@{}}
\toprule
\textbf{Bölge Adı} & \textbf{Konum} & \textbf{Etki ve Şiddet} & \textbf{Tarih} \\ \midrule
\textbf{Aladağ Yangını} & Safranbolu G, Aladağ & Orta-Yüksek Şiddet, ~1000 ha & 28 Tem - 2 Eyl \\
\textbf{Cumayanı (Cildikısık)} & Safranbolu K, Cumayanı & Düşük-Orta Şiddet, ~500 ha & 23-27 Tem \\
\textbf{Büyük Ovacık} & Ovacık, Beydini & Yüksek Şiddet (En Yıkıcı), ~6000 ha & 23-29 Tem \\
\textbf{Kışla Yangını} & Ovacık, Kışla Köyü & Orta Şiddet, ~400 ha & 23-26 Tem \\
\textbf{Soğuksu-Arıcak} & Merkez, Soğuksu TOKİ & Şehir Tehdidi, ~500 ha & 7-10 Ağu \\
\textbf{Toprakcuma} & Safranbolu-Kastamonu & Sınır Yayılımı, ~800 ha & 1-3 Eyl \\
\textbf{Eflani-Güzelce} & Eflani, Güzelce & Yüksek Şiddet, ~1500 ha & 31 Ağu - 3 Eyl \\ \bottomrule
\end{tabular}%
}
\end{table}

\subsubsection{Büyük Ovacık Yangını}
Sezonun en büyük yangını olan Ovacık yangını, Beydini ve Alınca köyleri çevresinde yoğunlaşmıştır. dNBR haritaları, bu bölgede bitki örtüsünün tamamen yok olduğu (High Severity) geniş alanları göstermektedir. 15'ten fazla köyün tahliye edilmesine neden olan bu yangın, ekolojik açıdan en büyük tahribatı yaratmıştır.

\subsubsection{Eflani ve Toprakcuma Yangınları}
İl sınırlarında (Kastamonu ve Çankırı) meydana gelen bu yangınlar, rüzgarın etkisiyle hızlı yayılım göstermiştir. Eflani-Güzelce yangınında dumanın Ankara'ya kadar ulaşması, yangının atmosferik etkisini de ortaya koymaktadır.

\section{Sonuç}
Bu çalışma, Sentinel-2 verilerinin orman yangınlarının izlenmesinde ve hasar tespitinde yüksek doğrulukla kullanılabileceğini göstermiştir. Karabük örneğinde elde edilen haritalar, yangın sonrası rehabilitasyon çalışmalarının önceliklendirilmesi (özellikle Büyük Ovacık ve Eflani bölgeleri) için altlık oluşturmaktadır.

\end{document}
